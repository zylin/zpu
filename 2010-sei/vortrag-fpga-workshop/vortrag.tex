\documentclass{beamer}
\usepackage[ngerman]{babel}
\usepackage[latin1]{inputenc}

% for flow chart
\usepackage{tikz}
\usetikzlibrary{shapes,arrows,trees,positioning}

\usetheme{default}

\usecolortheme{default}
\usefonttheme{default}
\useinnertheme{default}
\useoutertheme{default}

% Seitenzahlen in Fußzeile einfügen
\setbeamertemplate{fpptline}[frame number]

%
% Metainformationen
%
\title[]{hints \& tricks on FPGA projects}
\subtitle[]{}

\author[B. Lange]{Bert Lange}
\institute[FZD Dresden-Rossendorf]{Forschungszentrum Dresden-Rossendorf}
\date[29.09.10]{29. September 2010}
\logo{\pgfimage[width=10mm,height=10mm]{fzd_logo}}

\titlegraphic{\includegraphics[width=50mm]{fzd_logo_text}}

% pdf eigenschaften
\subject{}
\keywords{FPGA, Xilinx}

\begin{document}

%%%%%%%%%%%%%%%%%%%%%%%%%%%%%%%%%%%%%%%%%%%%%%%%%%%%%%%%%%%%%
\frame{
    \titlepage
}

\frame{
    \frametitle{Inhaltsverzeichnis}
    \tableofcontents
%    [pausesections]
}


%%%%%%%%%%%%%%%%%%%%%%%%%%%%%%%%%%%%%%%%%%%%%%%%%%%%%%%%%%%%%
\section{Designflow}
%
% 
%

\begin{frame}{Designflow}{Simulation \& Synthese}


\tikzstyle{block} = [draw, fill=blue!20, rectangle, rounded corners, 
    minimum height=2em, minimum width=6em]

\begin{columns}
    \column{.5\textwidth}

        \begin{tikzpicture}[auto, node distance=12mm, >=latex']

            \node [block, ] (ein) {Eingabe};
            \node [block, below of=ein, ] (syn) {Synthese};
            \node [block, below of=syn, ] (tst) {Test};

            \draw [->] (ein) -- (syn);
            \draw [->] (syn) -- (tst);
            \draw (tst) -| ++(2,1) [->] |- (ein);

        \end{tikzpicture}

    \column{.5\textwidth}
        
        \begin{tikzpicture}[auto, node distance=12mm, >=latex']

            \node [block, ] (ein) {Eingabe};
                \node [ right of=ein,] (end) {};
            \node [block, below of=ein, ] (sim) {Simulation};
            \node [block, below of=sim, ] (syn) {Synthese};
            \node [block, below of=syn, ] (tst) {Test};

            \draw [->] (ein) -- (sim);
            \draw (sim) -| ++(2,1) [->] |- (ein);
            \draw [->] (sim) -- (syn);
            \draw [->] (syn) -- (tst);
            %\draw (tst) -| ++(3,1) [->] |- ++(-1,2.6);
            \draw (tst) -| ++(3,1) [->] |- (end);

        \end{tikzpicture}

\end{columns}

\end{frame}

%%%%%%%%%%%%%%%%%%%%%%%%%%%%%%%%%%%%%%%%%%%%%%%%%%%%%%%%%%%%%
\section{Projektstruktur}

\begin{frame}{Projektstruktur}{}

\tikzstyle{block1} = [draw, fill=blue!20, rectangle, rounded corners, 
    minimum height=2em, minimum width=6em]
\begin{columns}
    \column{.5\textwidth}


        \begin{tikzpicture}[auto, node distance=15mm, >=latex']
        \node[block1, ] (tb) {Testbench
                \node[block1, ] (sys1) {System
                        \node[block1, ] (tb) {Module};
                };
        };
        \node[block1, ] (sys2) {System};
        \end{tikzpicture}

    \column{.5\textwidth}


\end{columns}

\end{frame}

%%%%%%%%%%%%%%%%%%%%%%%%%%%%%%%%%%%%%%%%%%%%%%%%%%%%%%%%%%%%%
\section{Typverwendung in Modulschnittstellen}

%%%%%%%%%%%%%%%%%%%%%%%%%%%%%%%%%%%%%%%%%%%%%%%%%%%%%%%%%%%%%
\section{numerische Bibliotheken}

\begin{frame}{Titel}{Untertitel}
    Inhalt
\end{frame}


\end{document}
